\documentclass{article}
\usepackage{hyperref}
\title{Homework 1 - CS352}
\author{Arun Felix}

\begin{document}
    \maketitle
    \tableofcontents
\newpage
\section{problem 1}

Consider the following circuit-switched network in Fig 1, and four switches A, B, C and D, going in the clockwise direction, and 4 circuits on each link. 

\begin{itemize}
    \item (a) In this network, calculate the maximum number of connections that can be in progress simultaneously at any one time? 
    \item (b) If all connections are between switches A and C, what is the maximum number of connections that can be in progress simultaneously?
    \item (c) If we want to have 4 connections between A and C, and
    another 4 connections between B and D. Can we route these calls through the 4 links to accommodate all 8 connections?
\end{itemize}

\subsection{part a}
In this network, we can only have 16 connections simultaneously.
\begin{itemize}
    \item 4 connections from A to B
    \item 4 connections from B to C
    \item 4 connections from A to D
    \item 4 connections from D to C
\end{itemize}

\subsection{part b}

Here, we can only have 8 connections:
\begin{itemize}
    \item 4 connections from $A \rightarrow B \rightarrow C$.
    \item 4 connections from $A \rightarrow D \rightarrow C$.
\end{itemize}

\subsection{part c}

We can indeed accomodate 4 connections between A and C, and another four between B and D.

\begin{itemize}
    \item 2 connections from $A \rightarrow B \rightarrow C$.
    \item 2 connections from $A \rightarrow D \rightarrow C$.
    \item 2 connections from $B \rightarrow C \rightarrow D$.
    \item 2 connections from $B \rightarrow A \rightarrow D$.
\end{itemize}

\section{Problem 2}

Consider the throughput example corresponding to Fig.2. If we have are M client-server pairs instead of 10. Denote $R_s$, $R_c$, and R for the rates of the server links, client links, and network link. If all other links have enough capacity and that there is no other traffic in the network besides the traffic generated by the M client-server pairs. Please derive a general expression for throughput in terms of $R_s$ , $R_c$ , R , and M.

\subsection{Response}

We want to generalize our figure to have
\begin{itemize}
    \item M client server pairs, and the links.
    \item $R_s$ for the rate of the server links.
    \item $R_c$ for the rate of the client link.
    \item $R$ for the rate of the network link.
\end{itemize}

The throughput will be bottlenecked by the transmission rate of the smallest transmission rate of the server, client and network link. since we have M client-server pairs and one network link, we can say the throughput is:
% \vspace{1em}
\begin{center}
    throughput = $min(MR_s, MR_c,R)$
\end{center}

\section{Problem 3}

Assume an HTTP client that wants to obtain a web document at a given URL. The IP address of the HTTP server is unknown at first. The web document at the URL has one embedded GIF image that resides at the same server as the original document. What transport and application layer protocols besides HTTP are needed in this scenario?

\subsection{Response}

\begin{itemize}
    \item We're using the HTTP protocol to access a web document.
    \item We first need to use a Domain Name System, which will find the IP address for the given URL.
    \item HTTP uses the TCP protocol to send the request message and receive the response message.
\end{itemize}

\section{Problem 4}

Suppose you click on a link to obtain a Web page in your Web browser. The IP address for the associated URL is not cached in your local host, so a DNS lookup is necessary to obtain this IP address. Assume that n DNS servers are visited before your host receives the IP address from DNS; the successive visits incur an RTT of $RTT_1$,...,$RTT_n$. Further assume that the Web page associated with the link contains exactly one object, consisting of a small amount of HTML text. Let $RTT_0$ denote the RTT between the local host and the server containing this object. Assuming zero transmission time of the object, how much time it will take from when the client clicks on the link until the client receives the object?

\subsection{Response}

Let's first think about the sequence of events occuring:

\begin{itemize}
    \item We need to look up the IP Address with the DNS, which is done n times. 
    \item Assuming we're using using HTTP, We will have to make a initial handshaking connection.
    \item Afterwards, we'll send all GET requests for all the files.
\end{itemize}

We can say the following regarding the total RTT.

\begin{itemize}
    \item we can say that the total visit time is $\sum_{i = 1}^{n} RTT_i$.
    \item We need to make a handshake between the server, which has a turnaround time of $RTT_0$
    \item We are only getting one object, hence the RTT is $RTT_0$.
    \item hence, the total time is $2RTT_0 + \sum_{i = 1}^{n} RTT_i$
\end{itemize}

\section{Problem 5}

Referring to the above problem, suppose this HTML file references 8 very small
objects on the same server. Neglecting transmission times, how much time elapses with

\begin{itemize}
    \item (i) Non-persistent HTTP with no parallel TCP connections?
    \item (ii) Non-persistent HTTP with the browser configured for 5 parallel connections?
    \item (iii) Persistent HTTP?
\end{itemize}

\subsection{Response}

Before answering each question, let's define some metrics:
\begin{itemize}
    \item The handshaking turnaround time is $T_0$ seconds.
    \item The turnaround time of the client requesting and receiving a file is $T_1$ seconds.
\end{itemize}

\subsubsection{Part i}
for non-parallel persistent http, we would need to do the following steps:
\begin{itemize}
    \item Establish a client-server connection.
    \item Client requests file from server.
\end{itemize}

Since it's nonpersistent, each connection can only handle one file. Since we have 8 files, we can say that the time it takes to get all 8 files would be $\sum_{i = 1}^{n} RTT_i + 16RTT_0$.

\subsection{Part ii}

We're doing nonpersistent http, but we have 5 parallel connections, so:

\begin{itemize}
    \item The client establish connections to the server for connections 1-5.
    \item The client will request and receive the file for connections 1-5.
    \item The client again establishes connections to the server for connections 6-8.
    \item The client will request and receive the file for connections 6-8.
\end{itemize}

This is much faster, as it'll take us $\sum_{i = 1}^{n} RTT_i + 4RTT_0$

\subsection{part iii}

Finally, we're using HTTP, which has the following steps:

\begin{itemize}
    \item the client establishes the connection, which takes $RTT_0$ time
    \item The client will send as many requests as it wants before closing (each file takes $RTT_0$ time.)
\end{itemize}

This will take $\sum_{i = 1}^{n} RTT_i + 9RTT_0$ seconds.
\end{document}
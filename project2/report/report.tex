\documentclass{article}

\begin{document}
\title{CS352 - Project 1 Report}
\author{Arun Felix and Bhavya Patel}
\maketitle

\section{Team Details}

\begin{itemize}
    \item Arun Felix (netid: ajf277)
    \item Bhavya Patel (netid: bsp75)
\end{itemize}

\section{Collaboration}

We used the Python documentation, and Bhavya focused on the first half. Arun did the last half and developed testcases.

\section{Implementation of recursive and iterative query functionality}

In this project, recursive resolution queries are implemented by having the client send a recursive query to the root server (RS), which then processes the query to determine if the domain is in its domain map. 
If not, the RS identifies the appropriate top-level domain (TLD) server (TS1 or TS2) based on the domain's TLD and forwards the query to that server using a recursive connection. 
The TLD server processes the query and sends back the resolved IP address or an indication that the domain is not found. 
The RS then forwards this response back to the client, completing the recursive resolution process.

\end{document}

